\input{../templates/course_definitions}
\input{../templates/course_information}
\usepackage[ngerman]{babel}
\usepackage{ulem}
\usepackage{csquotes}

\title{Java}
\subtitle{Einführung}
\date{\today}


\begin{document}

% \section{Organisation}
\begin{frame}
	\titlepage
\end{frame}
\begin{frame}{Überblick}
	\setbeamertemplate{section in toc}[sections numbered]
	\tableofcontents
\end{frame}

\section{Grundlegende Infos}
\begin{frame}{Kursinformationen}
% 	Language?\\
	Voraussetzungen
    \begin{itemize}[<+->]
        \item Du weißt, wie du deinen Computer bedienst 
        \item Du bringst deinen eigenen Laptop mit
        \item Du hast das JDK und eventuell die IDE deiner Wahl bereits eingerichtet
        \item Du hast eventuell schon Erfahrung mit anderen Programmiersprachen(nicht notwendig)
% 		\item Java is not your first programming language or you are a fast learner
	\end{itemize} 
    \onslide<+->{Weiteres}
    \begin{itemize}[<+->]
		\item Es wird \textasciitilde14 Termine geben
        \item Jeder Termin deckt \textasciitilde einen Themenbereich  ab
        \item Zu den meisten Themen gibt es Übungsaufgaben die Du zuhause bearbeiten kannst
	\end{itemize}
\end{frame}

% \subsection{Resources}
\begin{frame}{Ein paar nützliche Ressourcen}
    \begin{itemize}[<+->]
		\item Bei Problemen den Tutor fragen \hfill \\
            \url{Marcus.Koehler4@tu-dresden.de}
		\item Das Auditorium \hfill \\
			\url{http://auditorium.inf.tu-dresden.de}
		\item StackOverflow, FAQs, Online-tutorials, ... \hfill \\
		\item Offizielle Java-Dokumentation \hfill \\
			\url{https://docs.oracle.com/javase/10/}
        \item Programmierkurs-Mailinglist \url{programmierung@ifsr.de}
%        \item Cyberspace (wednesday 5./6. DS)
		\item Kurs-Repository \\
			\url{https://github.com/B4ckslash/java-lessons}
        \item Und natürlich, Google
	\end{itemize}
\end{frame}

\section{Java-Überblick}
\begin{frame}{Über Java}
	% \tikzoverlay at (7cm,1.4cm) {
	% 	\includegraphics[width=3cm, height=3cm]{res/logo-java.png}
	% }
	Pro:
	\begin{itemize}
		\item Syntax in der Tradition von ALGOL/C/C++
        \item Entworfen für objektorientiertes Programmieren(OOP)
		\item Plattformunabhängig (JVM)
        \item Automatisches Speichermanagement
		\item Keine zwingend nötigen externen Bibliotheken
        \item[] $->$ Einfache und \sout{problemlose} problemarme Verwendung\footnotemark[1]
	\end{itemize}
    \footnotetext[1]{Größtenteils}
\end{frame}

\begin{frame}{Über Java}
	% \tikzoverlay at (7cm,1.4cm) {
	% 	\includegraphics[width=3cm, height=3cm]{res/logo-java.png}
	% }
	Kontra:
	\begin{itemize}
        \item Viele features der JVM/des JDK werden nur \\ selten benötigt (teilweise mit Java 9 behoben\footnotemark[1])
		\item Langsamer als Maschinensprache
		\item Keine echte Mehrfachvererbung
		\item Schwache Generics
		\item Eher schlechte Unterstützung für \\ andere Programmierparadigmen
		\item[] $->$ Weder besonders schnell noch speicherfreundlich
	\end{itemize}
    \footnotetext[1]{Zumindest kann man die nicht benötigten Features aus einem custom JDK entfernen}
\end{frame}

\subsection{Ein erstes Programm}
\begin{frame}{Hello World}
  DEMO
\end{frame}

\begin{frame}[fragile]{Eine einfache Arbeitsumgebung(Linux/OS X)}
  In einem neuen Terminal:
    \begin{lstlisting}[language=bash]
      mkdir myProgram
      cd myProgram
      # touch Hello.java -- eigentlich nicht nötig
      vim Hello.java //spezifischer Editor ist egal\end{lstlisting}
\end{frame}

\begin{frame}[fragile]{Eine einfache Arbeitsumgebung(Windows)}
    \begin{itemize}[<+->]
        \item Methode 1: In einem neuen cmd-Fenster: \\
            \begin{lstlisting}[language=bash]
                mkdir myProgram
                cd myProgram
                notepad Hello.java\end{lstlisting}
        \item Methode 2: Im Explorer einen neuen Ordner + eine neue Datei anlegen und mit Notepad öffnen
    \end{itemize}
\end{frame}
\subsection{Hello World!}

\begin{frame}[fragile]{Hello World!}
	Eine leere Java-Klasse. \\
    Java-Klassen müssen immer mit einem Großbuchstaben beginnen.
    \begin{lstlisting}[gobble=4]
    public class Hello {

    }\end{lstlisting}
\end{frame}

\begin{frame}[fragile]{Hello World!}
    Dieses kurze Programm schreibt \texttt{Hello World!} auf die Konsole:
    \begin{lstlisting}[gobble=4]
    public class Hello {
        public static void main(String[] args) {
            System.out.println("Hello World!");
        }
    }\end{lstlisting}
\end{frame}

\begin{frame}[fragile]{Das Programm starten}
    Die Datei als Hello.java speichern \\ und dann folgende Befehle ausführen:
   \begin{lstlisting}[language=bash, gobble=8]
        javac Hello.java
        java Hello\end{lstlisting}
    Output:
    \begin{lstlisting}[gobble=8] 
        Hello World! \end{lstlisting}
\end{frame}

\subsection{Ein neues IntelliJ-Projekt erstellen}

\begin{frame}{Hello World in einer IDE}
   DEMO
\end{frame}

\section{Java-Basics}

\begin{frame}[fragile]{Kommentare}
    \begin{lstlisting}[gobble=4]
    public class Hello {
       // Outputs "Hello World!" to the console
       public static void main(String[] args) {
           System.out.println("Hello World!");
       }
    }\end{lstlisting}
   Idealerweise sollte man seinen Code immer kommentieren. \\
   Code wird oft häufiger gelesen als er geschrieben wird.
   \begin{itemize}
       \item \texttt{// Einzeiliger Kommentar}
       \item \texttt{/* Mehrzeiliger \\
           Kommentar*/}
   \end{itemize}
\end{frame}

\begin{frame}[fragile]{Kommentare}
    Ein paar Guidelines für Kommentare:
    \lstset{moredelim=[is][\sout]{|}{|}}
    \begin{itemize}[<+->]
        \item \enquote{Guter Code dokumentiert sich selbst}
            \begin{itemize}
                \item Aussagekräftige Namen wählen
                \item Komplexe Codeblöcke in kleine, einfache Teile aufbrechen \\ 
                    (-> Refactoring)
            \end{itemize}
        \item Keine offensichtlichen Infos wiederholen:
            \begin{lstlisting}[gobble=12]
            |// multipliziert y mit x|
            int z = y * x;\end{lstlisting}
    \end{itemize}
\end{frame}

\begin{frame}[fragile]{Kommentare}
    \begin{itemize}[<+->] 
        \item Notwendige, aber nicht intuitive Workarounds sollten erklärt werden:\footnotemark
            \begin{lstlisting}[gobble=12]
            function addSetEntry(set, value) {   
              /* 
                 Don't return `set.add` because it's not chainable in IE 11.
                   */  
                     set.add(value);    
                       return set;  
            }\end{lstlisting}
    \end{itemize}
    \footnotetext[1]{Code snippet: \url{https://lodash.com/docs}}
\end{frame}

\subsection{Definitionen}

\begin{frame}[fragile]{Code-Grundbausteine}
    \lstset{moredelim=[is][\textcolor{mymauve}]{|}{|}}
    \begin{lstlisting}[gobble=4]
	public class Hello {
	    // Calculates some stuff and outputs everything on the console
	    public static void main(String[] args) {	        
	        |int x;|
	        |x = 9;|
	        int y = 23;
	        int z;
	        z = x * y;
	        
	        |System.out.println(z);|
	    }
	}\end{lstlisting}
    Jede Deklaration, Zuweisung \& jeder Methodenaufruf ist ein \textit{Statement}
\end{frame}

\begin{frame}[fragile]{Code-Grundbausteine}
    Den vorherigen Code kann man leicht auf das folgende reduzieren:
    \begin{lstlisting}[gobble=4]
	public class Hello {
	    // Calculates some stuff and outputs everything on the console
	    public static void main(String[] args) {
	        System.out.println(9 * 23);
	    }
	}\end{lstlisting}
\end{frame}

\begin{frame}{Primitive Datentypen}
	Java unterstützt folgende primitive Datentypen:
    \begin{itemize}[<+->]
        \item\texttt{boolean}, einen Wahrheitswert (entweder \texttt{true} oder \texttt{false})
        \item\texttt{int}, eine 32-bit Ganzzahl(a.k.a. Integer)
        \item\texttt{long}, einen 64-bit Integer
        \item\texttt{float}, eine 32-bit Gleitkommazahl \\
            (engl. floating point number)
        \item\texttt{double}, eine 64-bit Gleitkommazahl
        \item\texttt{char}, ein 16-bit Unicode Zeichen \\ (engl. character)
        \item\texttt{void}, den leeren Typ (wird später gebraucht)
	\end{itemize}
    \onslide<+->{Es gibt außerdem noch zwei eher selten verwendete Datentypen:}
    \begin{itemize}
        \item<.-> \texttt{short}, einen 16-bit Integer
        \item<.-> \texttt{byte}, einen 8-bit Integer
    \end{itemize}
\end{frame}

\begin{frame}[fragile]{Blöcke}
    \lstset{moredelim=[is][\textcolor{red}]{|}{|}}
    \begin{lstlisting}[gobble=4]
    public class Hello |{|
        public static void main(String[] args) {
	    // prints a "Hello World!" on your console
            System.out.println("Hello World!");
	    }
	|}|\end{lstlisting}
    Alles zwischen \texttt{\{} und \texttt{\}} ist ein \textit{Block}. \\
	Blöcke können geschachtelt werden.
\end{frame}

\begin{frame}[fragile]{Das Semikolon}
    \lstset{moredelim=[is][\textcolor{red}]{|}{|}}
	\begin{lstlisting}[gobble=4]
	public class Hello {
	    // prints a "Hello World!" on your console
	    public static void main(String[] args) {
	        System.out.println("Hello World!")|;|
	    }
	}
	\end{lstlisting}
    Jedes \textit{Statement} wird von einem Semikolon abgeschlossen. \\
    \textit{Blöcke} müssen nicht mit einem Semikolon beendet werden.	
\end{frame}


\begin{frame}[fragile]{Variablennamen}
    \begin{itemize}[<+->]
		\item Variablennamen können mit einem beliebigen Buchstaben oder Unterstrich beginnen. \\
            Für gewöhnlich beginnen sie mit einem kleinen Buchstaben(abhängig von lokalen Konventionen und coding styles).
		\item Zusammengesetzte Namen sollten camelCase verwenden.
		\item Variablennamen sollten aussagekräftig sein.
	\end{itemize}
    \begin{onlyenv}<+->
        \begin{lstlisting}[gobble=4]
        public class Calc {
            public static void main(String[] args) {
                int a = 0; // not very meaningful
                float myFloat = 5.3f; // also not meaningful
                int count = 7; // quite a good name

                int rotationCount = 7; // there you go
            }
        }\end{lstlisting}
    \end{onlyenv}
\end{frame}

\subsection{Berechnungen}

\begin{frame}[fragile, allowframebreaks]{Berechnungen mit \texttt{int}}

    Java unterstützt folgende Operationen nativ: \\
    \smallskip
	\begin{tabular}{ll}
		Addition & \texttt{a + b;} \\
		Subtraktion & \texttt{a - b;} \\
		Multiplikation &\texttt{a * b;} \\
		Division & \texttt{a / b;} \\
        Modulo(Restdivision) & \texttt{a \% b;} \\
		Inkrementierung & \texttt{a++;} \\
		Dekrementierung & \texttt{a--;} \\
	\end{tabular} \\
    \smallskip
    Weitere arithmetische Operationen(Wurzel etc.) werden von Bibliotheken übernommen.
\framebreak
	\begin{lstlisting}[gobble=4]
	public class Calc {
	    public static void main(String[] args) {
	        int a; // declare variable a
	        a = 7; // assign 7 to variable a
	        System.out.println(a); // prints: 7
	        a = 8;
	        System.out.println(a); // prints: 8
	        a = a + 2;
	        System.out.println(a); // prints: 10
	    }
	}\end{lstlisting}
    Variablen mit primitiven Typen sind nach der ersten Zuweisung initialisiert.	
\framebreak
	\begin{lstlisting}[gobble=4]
	public class Calc {
	    public static void main(String[] args) {
	        int a = -9; // declaration and assignment of a
	        int b; // declaration of b
	        b = a; // assignment of b
	        System.out.println(a); // prints: -9
	        System.out.println(b); // prints: -9
	        a++; // increments a
	        System.out.println(a); // prints: -8
	    }
	}\end{lstlisting}
% \framebreak
% 	\begin{lstlisting}
% 	public class Calc {
% 	    public static void main(String[] args) {
% 	        int b; // declaration of b
% 	        System.out.println(b);
% 	    }
% 	}
% 	\end{lstlisting}
% 	Uninitialized variables will cause an Exception. \\
% 	An Exception is a kind of error we will discuss later.\\
% 	\vspace{1em}
% 	\emph{Always assign your variables!}
% \framebreak
\end{frame}

\begin{frame}[fragile, allowframebreaks]{Berechungen mit \texttt{float}}
    \begin{lstlisting}[gobble=4]
    public class Calc {
        public static void main(String[] args) {
            float a = 9;
            float b = 7.5f;
            System.out.println(a); // prints: 9.0
            System.out.println(b); // prints: 7.5
            System.out.println(a + b); // prints: 16.5
        }
    }\end{lstlisting}
\framebreak
	\begin{lstlisting}[gobble=4]
	public class Calc {
	    public static void main(String[] args) {
	        float a =       8.9f;
	        float b = 3054062.5f;
	        System.out.println(a); // prints: 8.9
	        System.out.println(b); // prints: 3054062.5
	        System.out.println(a + b); // prints: 3054071.5
	    }
	}\end{lstlisting}
	Gleitkommazahlen sind nicht beliebig genau. \\
	\emph{Dies kann manchmal zu unerwarteten und/oder falschen Ergebnissen führen!}
\end{frame}

\begin{frame}[fragile]{Arithmetik mit \texttt{int} und \texttt{float}}
	\begin{lstlisting}[gobble=4]
	public class Calc {
	    public static void main(String[] args) {
	        float a = 9.3f;
	        int b = 3;
	        System.out.println(a + b); // prints: 12.3
	        float c = a + b;
	        System.out.println(c); // prints: 12.3
            int d = a/c; // this shouldn't/doesn't work and should/will throw an Exception
	    }
	}\end{lstlisting}
    Java wandelt \texttt{int} automatisch in \texttt{float} um, falls nötig(\enquote{casting}). \\
	Umgekehrt aber nicht.
\end{frame}

\subsection{Text mit Strings}

\begin{frame}[fragile]{Strings}
    Ein String ist kein primitiver Datentyp, sondern ein Objekt\footnotemark[1]. \\
	Objekte werden in der nächsten Woche behandelt.
	\begin{lstlisting}[gobble=4]
	public class Calc {
	    public static void main(String[] args) {
	        String hello = "Hello World!"; // Strings sind immer von doppelten Anführungszeichen umgeben
	        System.out.println(hello); // print: Hello World!
	    }
	}\end{lstlisting}
    \footnotetext[1]{Deshalb beginnt der Datentyp auch mit einem Großbuchstaben}
\end{frame}

\begin{frame}[fragile]{Konkatenation}
	\begin{lstlisting}[gobble=4]
	public class Calc {
	    public static void main(String[] args) {
	        String hello = "Hello";
	        String world = " World!";
	        String sentence = hello + world;
	        System.out.println(sentence);
	        System.out.println(hello + " World!");
	    }
	}\end{lstlisting}
    Strings können mit \enquote{+} konkateniert(d.h. aneinandergehängt) werden. 
    Die ausgegebenen Zeilen sind gleich.
\end{frame}

\begin{frame}[fragile]{Strings und Zahlen}
	\begin{lstlisting}[gobble=4]
	public class Calc {
	    public static void main(String[] args) {
	    	int factorA = 3;
	    	int factorB = 7;
	    	int product = factorA * factorB;
	    	String answer =
	            factorA + " * " + factorB + " = " + product;
	        System.out.println(answer); // prints: 3 * 7 = 21
	    }
	}\end{lstlisting}
    Primitive Datentypen werden bei Konkatenation in ihren momentanen Wert als \texttt{String} umgewandelt\footnotemark[1].
    \footnotetext[1]{\scriptsize{Genau genommen ruft es die toString() Methode des zugehörigen Objekts auf, dazu nächste Woche mehr}}
\end{frame}

\begin{frame}[fragile]{Finde den Fehler}
    \begin{lstlisting}[gobble=8]
        public class calc {
            public static void main(string[] args) {
                int myInt = 5
                int mySecondInt;
                float myFloat = 7.4f;
                float result = myInt + myFloat
                result = result + mySecondInt;
                String out = Result = + result;
                System.out.println(out);
            };
        }\end{lstlisting}
\end{frame}

\begin{frame}[fragile]{Finde den Fehler}
    \lstset{moredelim=[is][\textcolor{red}]{|}{|}}
    \lstset{moredelim=[is][\textcolor{magenta}]{°}{°}}
    \begin{lstlisting}[gobble=8]
        public class |c|alc {
            public static void main(|s|tring[] args) {
                int myInt = 5°;°
                int mySecondInt;
                float myFloat = 7.4f;
                float result = myInt + myFloat°;°
                result = result + mySecondInt; //mySecondInt ist nicht initialisiert
                String out = °"°Result = °"° + result;
                System.out.println(out);
            }|;|
        }\end{lstlisting}
\end{frame}

\end{document}
