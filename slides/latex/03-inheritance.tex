\input{../templates/course_definitions}
\input{../templates/course_information}
\usepackage{csquotes}
\usepackage{ulem}


\title{Java}
\subtitle{Vererbung}
\date{\today}

\begin{document}

\begin{frame}
	\titlepage
\end{frame}

\begin{frame}{Overview}
	\setbeamertemplate{section in toc}[sections numbered]
	\tableofcontents
\end{frame}

\section{Datenkapselung}
\begin{frame}[fragile]{Zugriffsrechte}
    Um zu regeln, welche Objekte bzw. Klassen in Java worauf zugreifen dürfen, gibt es die folgenden Keywords:
    \begin{itemize}[<+->]
		\item \texttt{public}
		\item \texttt{private}
		\item \texttt{protected}
	\end{itemize}
\end{frame}
	
\begin{frame}[fragile]{Zugriffsrechte}
	\begin{lstlisting}[gobble=8]
		public class Student {
			public String getName() {
				return "Peter";
			}
			
			private String getFavouritePorn() {
				return "...";
			}
		}
	
		// [...]
		exampleStudent.getName(); // Works!
		exampleStudent.getFavouritePorn(); // Error
	\end{lstlisting}
\end{frame}

\begin{frame}[fragile]{Datenkapselung}
    \textit{Datenkapselung} bezeichnet das \enquote{Verstecken} der Attribute von Objekten. \\
    \begin{lstlisting}[gobble=8]
        public class Student{
            private String name;
            public int matNr;

            [...]
            public void setName(String name) {
                if(name.isEmpty() || name == null) return;
                this.name = name;
            }
        }
        [...]
            Student s = new Student("Max", 4567283);
            s.name = ""; // throws exception
            s.setName(""); //doesn't change name
            s.matNr = -1; // works, but makes no sense
            s.setName("Peter"); //works
        [...]
    \end{lstlisting}
\end{frame}

\section{Mehr über Arrays}
\subsection{Mehrdimensionale Arrays}
\begin{frame}[fragile]{2-Dimensional Array}
	Arrays können mehrdimensional sein. \\
    Ein n-dimensionales Array hat dementsprechend auch n Indizes.
    \begin{lstlisting}[gobble=8]
        public static void main(String[] args) {

            // an array with 100 elements
            int[][] intArray = new int[10][10];
            
            intArray[0][0] = 0;
            intArray[0][9] = 9;
            intArray[9][9] = 99;
        }
	\end{lstlisting}
\end{frame}

\begin{frame}[fragile]{Arrays von Objekten}
    Es ist natürlich auch möglich, ein Array von Objekten anzulegen:
    \begin{lstlisting}[gobble=8]
        public static void main(String[] args) {

            Student[][] studentArray = new Student[10][10];
            
            for(int i = 0; i < 10; i++) {
                for(int j = 0; j < 10; j++) {
                    studentArray[i][j] = new Student();
                }
            }
        }
	\end{lstlisting}
\end{frame}

\section{Vererbung}
\subsection{Vererbung}
\begin{frame}[fragile]{Eine Lieferung}
    Die folgende Klasse \texttt{Letter} ist eine Unterklasse von \texttt{Delivery}, erkennbar am Keyword \texttt{extends}.
	\begin{itemize}
        \item \texttt{Letter} ist eine \textit{Unterklasse} bzw \textit{Subklasse} der Klasse \texttt{Delivery}
		\item \texttt{Delivery} ist die \textit{Superklasse} von \texttt{Letter}
	\end{itemize}
    \begin{lstlisting}[gobble=8]
        public class Letter extends Delivery {
        
        }
	\end{lstlisting}
	\vfill
    \pause
	Eine Klasse kann mehrere Subklassen haben. \\
    Umgekehrt kann eine Klasse maximal eine Superklasse haben. \\
    Eine Subklasse \textit{erbt} von ihrer Superklasse.
\end{frame}

\begin{frame}[fragile]{Beispiel}
    Für heute haben wir die Klassen \texttt{PostOffice}, \texttt{Delivery} and \texttt{Letter}.
    Im Verlauf des Kurses werden sie immer weiter wachsen.
	\begin{lstlisting}
        public class Delivery {
            private String address;
            private String sender;
            
            public void setAddress(String addr) {
                address = addr;
            }
            
            public void setSender(String snd) {
                sender = snd;
            }
            
            public void printAddress() {
                System.out.println(this.address);
            }
        }
	\end{lstlisting}
\end{frame}

\begin{frame}[fragile]{Methodenvererbung}
	Mithilfe von Vererbung kann man auf einer Instanz von \texttt{Letter} die Methoden von \texttt{Delivery} aufrufen:
    \begin{lstlisting}[gobble=8]
        public class PostOffice {
        
            public static void main(String[] args) {
            
                Letter letter = new Letter();
                
                letter.setAddress("cafe ascii, Dresden");
                
                letter.printAddress();
                // prints: cafe ascii, Dresden
            }	
        }
	\end{lstlisting}
\end{frame}

\begin{frame}[fragile]{Methoden-Override}
    Im folgenden Code-Snippet wird die Methode printAdress() in der Klasse \texttt{Letter} neu definiert:
    \begin{lstlisting}[gobble=8]
        public class Letter extends Delivery {
            @Override
            public void printAddress() {
                System.out.println("a letter for " + this.address);    
            }	
        }
	\end{lstlisting}
    \pause
    \texttt{@Override} ist eine \textit{Annotation}.
    Annotationen sind nicht zwingend notwendig, machen den Code allerdings leichter verständlich.
    Wofür Annotationen sonst noch gebraucht werden, besprechen wir in kommenden Kursen.
\end{frame}
\begin{frame}[fragile]{Methoden-Override}
	Durch den Override wird jetzt die \texttt{printAddress()}-Methode aus der Klasse \texttt{Letter} verwendet.
    \begin{lstlisting}[gobble=8]
        public class PostOffice {
            public static void main(String[] args) {
                Letter letter = new Letter();
                
                letter.setAddress("cafe ascii, Dresden");
                
                letter.printAddress();
                // prints: a letter for cafe ascii, Dresden
            }	
        }
	\end{lstlisting}
\end{frame}

\subsection{Konstruktor}
\begin{frame}[fragile]{Super()}
	Falls in der Superklasse(hier \texttt{Delivery}) einen \textbf{Konstruktor mit Parametern} definieren, \\
    müssen wir das auch in jeder Unterklasse tun.
    \begin{lstlisting}[gobble=8]
        public class Delivery {
            private String address;
            private String sender;
            
            public Delivery(String address, String sender) {
                this.address = address;
                this.sender = sender;
            }
                    
            public void printAddress() {
                System.out.println(address);
            }
        }
	\end{lstlisting}
\end{frame}

\begin{frame}[fragile]{\texttt{super()}}
	Innerhalb des Konstruktors von \texttt{Letter} können wir mittels des Aufrufs \texttt{super()} den Konstruktor der Superklasse aufrufen. 
	\begin{lstlisting}[gobble=8]
        public class Letter extends Delivery {
            public Letter(String address, String sender) {
                super(address, sender);
            }
        
            @Override
            public void printAddress() {
                System.out.println("a letter for " + this.address);    
            }	
        }
	\end{lstlisting}
\end{frame}

\begin{frame}[fragile]{Super() - Test}
    \begin{lstlisting}[gobble=8]
        public class PostOffice {
            public static void main(String[] args) {	    
                Letter letter = new Letter("cafe ascii, Dresden", "");
                
                letter.printAddress();
                // prints: a letter for cafe ascii, Dresden
            }
        }
	\end{lstlisting}
\end{frame}

\subsection{Implizite Vererbung}
\begin{frame}{java.lang.Object}
	Jede Klasse in Java erbt automatisch von \texttt{java.lang.Object}. 
	Deshalb erbt jede Klasse die Methoden von \texttt{Object}.
	\vfill
    Alle Methoden von \texttt{Object} sind unter \scriptsize\url{http://docs.oracle.com/javase/10/docs/api/java/lang/Object.html} \normalsize \\ zu
	finden.
\end{frame}

\begin{frame}[fragile]{toString()}
	\texttt{Letter}ist eine Unterklasse von \texttt{Object}.
	Deshalb erbt \texttt{Letter} die Methode \texttt{toString()} von \texttt{Object}.\\
	\texttt{System.out.println(argument)} ruft \texttt{argument.toString()} auf, um einen String zu erhalten, der ausgegeben werden kann.
    \begin{lstlisting}[gobble=8]
        public class PostOffice {
            public static void main(string[] args) {	    
                letter letter = 
                    new letter("cafe ascii, dresden", "");
                
                system.out.println(letter);
                // prints: letter@_some_hex-value_
                // for example: letter@4536ad4d
            }
        }
	\end{lstlisting}
\end{frame}

\begin{frame}[fragile]{toString()-Override}
    \begin{lstlisting}[gobble=8]
        public class Letter extends Delivery {
            public letter(string address, string sender) {
                super(address, sender);
            }
        
            @Override
            public string tostring() {
                return "a letter for " + this.address;
            }	
        }
	\end{lstlisting}
\end{frame}

\begin{frame}[fragile]{toString()-Override - Test}
    \begin{lstlisting}[gobble=8]
        public class PostOffice {
            public static void main(String[] args) {	    
                Letter letter = 
                    new Letter("cafe ascii, Dresden", "");
                
                System.out.println(letter);
                // a letter for cafe ascii, Dresden
            }
        }
	\end{lstlisting}
\end{frame}

\subsection{Praktisches Beispiel}
\begin{frame}{Beispiel}
        \center{-> IntelliJ}
\end{frame}

\end{document}
