\input{../templates/course_definitions}
\input{../templates/course_information}
\usepackage{csquotes}
\usepackage{qtree}

\title{Java}
\subtitle{\texttt{abstract} \& Exceptions}
\date{\today}

\begin{document}

\begin{frame}
\titlepage
\end{frame}

\begin{frame}{Überblick}
    \setbeamertemplate{section in toc}[sections numbered]
    \tableofcontents
\end{frame}

\section{\texttt{abstract}}
\subsection{Intro}
\begin{frame}[fragile]{Abstrakte Klassen}
    \textit{Abstrakte Klassen} repräsentieren eine \enquote{Mischung} aus Interfaces und Vererbung,
    da sie sowohl Methoden \& Konstruktoren als auch \enquote{Vorschriften}(Methodensignaturen) enthalten kann. \\
    Abstrakte Klassen sind durch das Keyword \texttt{abstract} gekennzeichnet.
    \begin{lstlisting}[gobble=8]
        public abstract class AbstractExample {

        }
    \end{lstlisting}
    \pause
    \begin{itemize}[<+->]
        \item Abstrakte Klassen sind gut geeignet um ein \enquote{Interface} mit vielen \texttt{default}-Methoden zu realisieren.
        \item Von einer abstrakten Klasse können keine Objekte erzeugt werden.
        \item Abstrakte Klasse können von anderen abstrakten Klassen erben und können interfaces implementieren.
        \item Normale und abstrakte Klassen können von abstrakten Klassen erben.
    \end{itemize}
\end{frame}

\begin{frame}[fragile]{Methoden}
    Eine abstrakte Klasse kann sowohl \enquote{normale} Methoden als auch \textit{abstrakte} Methoden haben:
    \begin{lstlisting}[gobble=8]
        public abstract class AbstractExample {
            public void printHello() {
                System.out.println("Hello");
            }

            public abstract String getName();
        }
    \end{lstlisting}
    Abstrakte Methoden bestehen(wie bei Interfaces) nur aus der Methodensignatur(und dem Keyword \texttt{abstract}).
    Sobald man eine abstrakte Methode deklariert, muss die enthaltende Klasse ebenfalls als abstrakt markiert werden.
\end{frame}

\begin{frame}[fragile]{Vererbung}
    Jede erbende Klasse einer abstrakten Klasse muss entweder alle abstrakten Methoden implementieren oder selbst abstrakt sein.
    Alle \enquote{normalen} Methoden werden wie gewohnt übernommen.
    \begin{lstlisting}[gobble=8]
        public class Example extends AbstractExample {
            @Override
            public String getName() {
                return "Example";
            }
        }
    \end{lstlisting}
\end{frame}

\section{Exceptions}
\subsection{Überblick}
\begin{frame}{Exceptions}
    Idealerweise würden bei der Ausführung eines Programms nie Fehler oder generell unerwartetes Verhalten auftreten.
    Da die Welt allerdings nicht ideal ist, muss man sich als Programmierer (leider) mit solchen Situationen auseinandersetzen.

    Java unterstützt Fehlerbehandlung nativ, mit sogenannten \textit{Exceptions}.
    Exceptions separieren Fehlersituationen und -behandlung von \enquote{normalem} Code.
\end{frame}

\begin{frame}{Hierarchie}
    \center{\Tree [.Object [.Throwable Error [.Exception \dots{} RuntimeException ] ] ]} \\
    \smallskip
    Jede Exception ist eine Subklasse von \texttt{Throwable}.
    \texttt{Error} ist ebenfalls eine Subklasse von \texttt{Throwable}, ist aber für schwerwiegende Fehler innerhalb der JVM bzw. des JRE reserviert.
    Da jede Exception auch \enquote{throwable} ist, sagt man auch, dass Exceptions \textit{geworfen} werden.

    \scriptsize\url{https://docs.oracle.com/javase/10/docs/api/java/lang/Throwable.html}
\end{frame}

\begin{frame}{(Un-)Checked Exceptions}
    Alle Exceptions außer \texttt{RuntimeException} und ihrer Subklassen sind sogenannte \textit{checked Exceptions}.
    \pause
    \begin{itemize}[<+->]
        \item Checked exceptions müssen entweder sofort behandelt oder weitergegeben werden(dazu später mehr)
        \item Die Ursache solcher Exceptions liegt meistens außerhalb des Programms(z.B. Dateien oder Datenbanken)
    \end{itemize}
    \onslide<+->{\texttt{RuntimeException} und ihre Subklassen dagegen sind sogenannte \textit{unchecked Exceptions}.}
    \begin{itemize}[<+->]
        \item Unchecked Exceptions müssen nicht zwingend behandelt oder weitergegeben werden.
        \item Solche Exceptions treten meistens aufgrund eines Fehlers im Programmcode auf, wie z.B. eine NullPointerException.
        \item Prinzipiell kann jede Methode eine solche Exception erzeugen.
    \end{itemize}
\end{frame}

\subsection{Exceptions behandeln}
\begin{frame}[fragile]{Intro}
    \begin{lstlisting}[gobble=8]
        public class Calc {

            public static void main(String[] args) {

                int a = 7 / 0;
                // will cause an ArithmeticException

                System.out.println(a);
            }
        }
    \end{lstlisting}
    Division durch 0 wirft eine \texttt{ArithmeticException}, was eine Subklasse von \texttt{RuntimeException} ist.
    Deshalb ist \texttt{ArithmeticException} eine unchecked Exception und muss nicht zwingend behandelt werden.
\end{frame}

\begin{frame}[fragile]{\texttt{try} und \texttt{catch}}
    Sobald eine Exception auftritt, stürzt das Programm ab, sofern sie nicht behandelt wird.
    Auch wenn eine Exception nicht behandelt werden muss, kann man sie trotzdem mittels eines \texttt{try-catch}-Blocks behandeln:
    \begin{lstlisting}[gobble=8]
        public class Calc {
            public static void main(String[] args) {

                try {
                    int a = 7 / 0;
                } catch (ArithmeticException e) {
                    System.out.println("Division by zero.");
                }
            }
        }
    \end{lstlisting}
    Der \texttt{catch}-Block, auch \textit{Exception handler} genannt,
    wird aufgerufen, wenn die angegebene Exception im \texttt{try}-Block auftritt.\\
    Man kann innerhalb eines \texttt{catch}-Blocks auch mehrere Exceptions abfangen.
\end{frame}

\begin{frame}[fragile]{\texttt{try} und \texttt{catch}}
    Man kann (sollte aber nicht!) jede mögliche Exception mit einem \texttt{try-catch} abfangen:
    \begin{lstlisting}[gobble=8]
        try{
            ...
        } catch (Exception e) { ... }
    \end{lstlisting}
    Auch wenn es erlaubt ist, geht bei solchen Konstrukten sehr schnell verloren, was der genaue Fehler war.
\end{frame}

\begin{frame}[fragile]{Stacktraces}
    \begin{lstlisting}[gobble=8]
        public class Calc {
            public static void main(String[] args) {

                try {
                    int a = 7 / 0;
                } catch (ArithmeticException e) {
                    System.out.println("Division by zero.");
                    e.printStackTrace();
                }
            }
        }
    \end{lstlisting}
    Ein \textit{Stacktrace} gibt alle Methodenaufrufe(d.h. den \textit{call-stack}) wieder, die zum Zeitpunkt der Exception aktiv waren.
\end{frame}

\begin{frame}[fragile]{Stack Trace}
    \begin{lstlisting}[gobble=8]
        Division by zero.
        java.lang.ArithmeticException: / by zero
            at Calc.main(Calc.java:6)
            at sun.reflect.NativeMethodAccessorImpl.invoke0(Native Method)
            at sun.reflect.NativeMethodAccessorImpl.invoke(NativeMethodAccessorImpl.java:62)
            at sun.reflect.DelegatingMethodAccessorImpl.invoke(DelegatingMethodAccessorImpl.java:43)
            at java.lang.reflect.Method.invoke(Method.java:498)
            at com.intellij.rt.execution.application.AppMain.main(AppMain.java:147)
    \end{lstlisting}
\end{frame}

\begin{frame}[fragile]{\texttt{try-catch-finally}}
    Bei einigen Ressourcen(z.B. Dateien) ist es wichtig, dass bestimmte beendende Methoden auch dann aufgerufen werden, wenn etwas schiefgeht.
    Hierzu gibt es den \texttt{finally}-Block:
    \begin{lstlisting}[gobble=8]
        public class Calc {
            public static void main(String[] args) {
                try {
                    int a = 7 / 0;
                } catch (ArithmeticException e) {
                    System.out.println("Division by zero.");
                    e.printStackTrace();
                } finally {
                    System.out.println("End of program.");
                }
            }
        }
    \end{lstlisting}
    Ein \texttt{finally}-Block wird immer ausgeführt, egal ob tatsächlich eine Exception auftritt oder nicht.
\end{frame}

\subsection{Exceptions werfen}
\begin{frame}[fragile]{Exceptions weitergeben}
    Unhandled exceptions können auch wieder geworfen(weitergegeben) werden.
    Dies wird mithilfe des Keywords \texttt{throws} am Ende der Methodensignatur getan:
    \begin{lstlisting}[gobble=8]
        public static int divide (int dividend, int divisor) throws ArithmeticException {
            return dividend / divisor;
        }
    \end{lstlisting}
    Durch das Keyword \texttt{throws} wird jede Exception, die den passenden Typ hat und
    innerhalb der Methode \texttt{\textcolor{blue}{int} divide(\dots)} auftritt an die aufrufende Methode weitergegeben.
\end{frame}

\begin{frame}[fragile]{Exceptions weitergeben - Test}
    \begin{lstlisting}[gobble=8]
        public class Calc {
            public static int divide (int dividend, int divisor) throws ArithmeticException {
                return dividend / divisor;
            }

            public static void main(String[] args) {

                int a = 0;
                try {
                    a = Calc.divide(7, 0);
                } catch (ArithmeticException e) {
                    System.out.println("Division by zero.");
                    e.printStackTrace();
                }
            }
        }
    \end{lstlisting}
\end{frame}

\begin{frame}[fragile]{Exceptions weitergeben - Test}
    \begin{lstlisting}[gobble=8]
        public static void main(String[] args) {
            int a = 0;
            try {
                a = Calc.divide(7, 0);
            } catch (ArithmeticException e) {
                System.out.println("Division by zero.");
                e.printStackTrace();
            }
        }
    \end{lstlisting}
    In diesem Beispiel ist eine neue Methode(die aufrufende Methode) zum Stacktrace dazu gekommen:
    \texttt{java.lang.ArithmeticException: / by zero}\\
    \texttt{at Calc.divide(Calc.java:4)}\\
    \texttt{at Calc.main(Calc.java:11)}
\end{frame}

\begin{frame}[fragile]{Eigene Exceptions}
    Wie auch sonst alles in Java, kann man auch seine eigenen Exceptions erstellen und nutzen:
    \begin{lstlisting}[gobble=8]
        public class DivisionByZeroException extends Exception {
        }
    \end{lstlisting}
    \vfill
    \begin{lstlisting}[gobble=8]
        public static int divide (int dividend, int divisor) throws DivisionByZeroException {
            if (divisor == 0) {
                throw new DivisionByZeroException();
            }
            return divident / divisor;
        }
    \end{lstlisting}
    Exceptions können \enquote{von Hand} mit dem Keyword \texttt{throw} geworfen werden.
\end{frame}

\begin{frame}[fragile]{Eigene Exceptions - Test}
    \begin{lstlisting}[gobble=8]
        public static void main(String[] args) {
            int a = 0;
            try {
                a = Calc.divide(7, 0);
            } catch (DivisionByZeroException e) {
                System.out.println("Division by zero.");
                e.printStackTrace();
            }
        }
    \end{lstlisting}
    Da \texttt{DivisionByZeroException} direkt von \texttt{Exception} erbt, ist sie eine checked Exception.
\end{frame}

\begin{frame}{Übungsaufgabe}
    \center
    \large{Such dir eine oder zwei Exceptions aus der Java Reference und schreibe Code, in dem sie auftritt. \\
    Füge dann den Code hinzu, der diese Exception a) behandelt und den Stacktrace ausgibt und b) sie an die aufrufende Methode weitergibt.}
\end{frame}

\end{document}
