\input{../templates/course_definitions}
\input{../templates/course_information}
\usepackage{csquotes}

\title{Java}
\subtitle{Generics \& die Collections-API}
\date{\today}

\begin{document}

\begin{frame}
\titlepage
\end{frame}

\begin{frame}{Überblick}
    \setbeamertemplate{section in toc}[sections numbered]
    \tableofcontents
\end{frame}

\section{Generics}
\subsection{Was sind Generics?}
\begin{frame}[fragile]{Generics?}
    \lstset{escapeinside={(*@)}{(@*)}}
    Was tun, wenn man Datenstrukturen wie Graphen und Listen auf einen bestimmten Typen beschränken will,
    aber nicht für jede denkbare Kombination aus Typen und Datenstruktur neuen Code schreiben will?
    \begin{lstlisting}[gobble=8,basicstyle=\ttfamily\scriptsize]
        public class StringTree {
            public void add(String item) {}
            (*@)\dots(@*)
        }
        public class IntegerTree {
            public void add(int item) {}
             (*@)\dots(@*)
        }
        (*@)\dots(@*)
    \end{lstlisting}
\end{frame}

\begin{frame}[fragile]{Generics?}
    \lstset{escapeinside={(*@)}{(@*)}}
    Eine mögliche Antwort darauf ist Polymorphie.
    Allerdings hat man dabei das Problem, dass man schnell in komplexes casting rutscht,
    wenn man mehr als die Methoden des gemeinsamen \enquote{Vorfahren} der zu verwendenden Typen benutzen will:
    \begin{lstlisting}[gobble=8,basicstyle=\ttfamily\scriptsize]
        public class Tree {
            public void add(Object item) {(*@)\dots(@*)}
            public Object get(int id) {(*@)\dots(@*)}
        }
        (*@)\dots(@*)
        Tree tree = new Tree();
        tree.add("hello"); //works because any String is also an Object
        String s = (String)tree.get(0); //must be cast because get(...) returns an Object
    \end{lstlisting}
    \pause
    Hierbei passiert es schnell, dass man inkompatible Typen wie \texttt{String} und \texttt{int} vermischt und ein Fehler beim casting auftritt.
    Außerdem würde solcher Code ohne Fehler kompilieren, weshalb man nicht immer sofort bemerkt, wenn eine solche Situation auftritt.
\end{frame}

\begin{frame}[fragile]{Generics!}
    Um sogenannte \textit{compile-time safety} zu garantieren, hat Java für solche Probleme die \textit{Generics} eingeführt. \\
    Mit Generics kann \enquote{automatisch} der Typ von Attributen, Parametern und Rückgabewerten angepasst werden:
    \begin{lstlisting}[gobble=8,basicstyle=\ttfamily\scriptsize]
        public class Box<T> {
            // T stands for "Type"
            private T t;

            public void set(T t) { this.t = t; }
            public T get() { return t; }
        }

        Box<Integer> integerBox = new Box<Integer>();
    \end{lstlisting}
    Man kann auch mehr als ein Generic pro Klasse verwenden:
    \begin{lstlisting}[gobble=8,basicstyle=\ttfamily\scriptsize]
        public class Dictionary<K,V> {
            //K = Key, V = Value
            ...
            public V getByKey(K key) { ... };
        }
    \end{lstlisting}
\end{frame}

\subsection{Wrapper-Klassen}
\begin{frame}{Wrapper-Klassen}
    Primitive Datentypen können nicht als Typen von Generics auftreten. \\
    Um primitive Datentypen trotzdem zu verwenden, muss man die folgenden \textit{Wrapper-Klassen} verwenden:
    \begin{center}
        \begin{tabular}{ c  c }
            boolean & Boolean \\
            byte & Byte \\
            char & Character \\
            int & Integer \\
            float & Float \\
            double & Double \\
            long & Long \\
            short & Short
        \end{tabular}
    \end{center}
\end{frame}

\section{Collections}
\subsection{Intro}
\begin{frame}{Die Collections-API}
    Java hat einige simple Datenstrukuren wie \texttt{Sets}, \texttt{Lists} und \texttt{Maps}.
    Diese Interfaces gehören zur sogenannten \textit{Collections-API}.

    Das \textit{Package} \texttt{java.util} enthält neben den o.g. Interfaces auch verschiedene Implementierungen davon,
    wie z.B. \texttt{ArrayList}, \texttt{HashMap} etc.
    Alternativ kann man auch seine eigenen Implementierungen schreiben und verwenden.
\end{frame}

\subsection{Set und List}
\begin{frame}[fragile]{\texttt{java.util.Set}}
    Ein \texttt{Set} aus dem \texttt{java.util}-Package ist eine Menge von Elementen des gleichen Typs, die kein Element zweimal enthalten kann.
    \begin{lstlisting}[gobble=8]
        import java.util.Set;
        import java.util.HashSet;

        public class SetTest {
            public static void main(String[] args) {
                Set<String> set = new HashSet<String>();

                set.add("foo");
                set.add("bar");
                set.add("bar"); //throws IllegalMethodException
                set.remove("foo");
                System.out.println(set); // prints: [bar]
            }
        }
    \end{lstlisting}
\end{frame}

\begin{frame}[fragile]{List}
    Eine \texttt{java.util.List} ist eine geordnete Menge von Objekten. \\
    Eine \texttt{LinkedList} ist in Java als eine doubly-linked-list realisiert.
    \begin{lstlisting}[gobble=8]
        import java.util.List;
        import java.util.LinkedList;

        public static void main(String[] args) {
            List<String> list = new LinkedList<String>();

            list.add("foo");
            list.add("foo"); // insert "foo" at the end
            list.add("bar");
            list.add("foo");
            list.remove("foo"); // removes the first "foo"

            System.out.println(list); // prints: [foo, bar, foo]
        }
    \end{lstlisting}
\end{frame}

\begin{frame}[fragile]{Listen-Methoden}
    Einige oft verwendete Methoden von \texttt{List<E>}: \\
    \smallskip
    \begin{tabular}{ r l r }
        void & \scriptsize{\texttt{add(int index, E element)}} & \footnotesize{Element an Indexposition einfügen} \\
        E &\scriptsize{\texttt{get(int index)}} & \footnotesize{Element an gegebenem Index ausgeben} \\
        E &\scriptsize{\texttt{set(int index, E element)}} & \footnotesize{Element an Indexposition ersetzen} \\
        E &\scriptsize{\texttt{remove(int index)}} & \footnotesize{Element an Indexposition entfernen}
    \end{tabular}
    \vfill
    Spezifische Methoden von \texttt{LinkedList<E>}: \\
    \smallskip
    \begin{tabular}{ r l r }
        void & \scriptsize{\texttt{addFirst(E element)}} & \footnotesize{Element am Anfang einfügen} \\
        E & \scriptsize{\texttt{getFirst()}} & \footnotesize{Erstes Element ausgeben} \\
        void & \scriptsize{\texttt{addLast(E element)}} & \footnotesize{Element am Ende einfügen} \\
        E & \scriptsize{\texttt{getLast()}} & \footnotesize{Letztes Element ausgeben}
    \end{tabular}
\end{frame}

\subsection{Iterieren}
\begin{frame}[fragile]{\texttt{for-each}}
    Java hat eine spezielle Variante der \texttt{for}-Schleife für Collections, die \texttt{for-each}-Schleife: \\
    \texttt{for (E e : collection)}
    \begin{lstlisting}[gobble=8]
        public static void main(String[] args) {
            List<Integer> list = new LinkedList<Integer>();

            list.add(1);
            list.add(3);
            list.add(3);
            list.add(7);

            for (Integer i : list) {
                System.out.print(i + " "); // prints: 1 3 3 7
            }
        }
    \end{lstlisting}
    \enquote{Hinter den Kulissen} verwendet eine solche Schleife einen \texttt{Iterator}.
\end{frame}

\begin{frame}[fragile]{\texttt{Iterator}}
    Ein \texttt{Iterator} geht Element für Element über eine Collection:
    \begin{lstlisting}[basicstyle=\ttfamily\scriptsize,gobble=8]
        public static void main(String[] args) {
            List<Integer> list = new LinkedList<Integer>();

            list.add(1);
            list.add(3);
            list.add(3);
            list.add(7);

            Iterator<Integer> iter = list.iterator();
            while (iter.hasNext()) {
                System.out.print(iter.next());
            }
            // prints: 1337
        }
    \end{lstlisting}
\end{frame}

\begin{frame}[fragile]{\texttt{Iterator}}
    Ein \enquote{normaler} \texttt{Iterator} hat die folgenden Methoden:
    \begin{itemize}
        \item \texttt{boolean hasNext()} - Gibt an, ob es noch Elemente nach dem momentanen Element gibt
        \item \texttt{E next()} - Gibt das nächste Element zurück(\textit{kann nur einmal pro Element aufgerufen werden!})
        \item \texttt{void remove()} - Entfernt das momentane Element(Wird nicht immer unterstützt)
    \end{itemize}
    Ein \texttt{Iterator} wird mittels \texttt{collection.iterator()} erzeugt:
    \begin{lstlisting}[basicstyle=\ttfamily\scriptsize,gobble=8]
        Collection<E> collection = new Implementation<E>;
        Iterator<E> iter = collection.iterator();
    \end{lstlisting}
\end{frame}

\begin{frame}[fragile]{\texttt{Iterator}-Workflow}
    \begin{semiverbatim}
         Collection: \indent e1\indent e2\indent \dots\indent eN
        Iter.-Index: \onslide<2>{\textasciicircum} \indent \onslide<3>{\textasciicircum} \indent \onslide<4>{\textasciicircum} \indent\onslide<5>{\textasciicircum} \indent \onslide<6>{\textasciicircum}
    \end{semiverbatim}
    \begin{enumerate}[<+->]
        \item Collection anlegen
        \item \texttt{Iterator} mittels \texttt{collection.iterator()} erzeugen
        \item \texttt{Iterator.next()} aufrufen(gibt \texttt{e1} zurück); \texttt{Iterator.hasNext() == true}
        \item \texttt{Iterator.next()} aufrufen(gibt \texttt{e2} zurück); \texttt{Iterator.hasNext() == true}
        \item \dots
        \item \texttt{Iterator.next()} aufrufen(gibt \texttt{eN} zurück); \texttt{Iterator.hasNext() == false}
    \end{enumerate}
\end{frame}

\subsection{Map}
\begin{frame}[fragile]{\texttt{java.util.Map}}
    Anders als die bisher vorgestellten Interfaces ist \texttt{java.util.Map} kein \enquote{Subinterface} von \texttt{java.util.Collection}. \\
    In einer \texttt{Map} sind Schlüssel und Werte gespeichert, wobei jeder Schlüssel einzigartig ist und jedem Schlüssel genau ein Wert zugeordnet ist.
    Werte können jedoch mehrfach vorkommen, weshalb zwei Schlüsseln der gleiche Wert zugeordnet sein kann.
    \begin{lstlisting}[gobble=8]
        public static void main (String[] args) {
            Map<Integer, String> map = new HashMap<Integer, String>();

            map.put(23, "foo");
            map.put(28, "foo");
            map.put(31, "bar");
            map.put(23, "bar"); // "bar" replaces "foo" for key = 23

            System.out.println(map);
            // prints: {23=bar, 28=foo, 31=bar}
        }
    \end{lstlisting}
\end{frame}

\begin{frame}[fragile]{\texttt{keySet()} \& \texttt{values()}}
    Da die Schlüssel einer \texttt{Map} einzigartig sind, kann man sie mittels \texttt{keySet()} als \texttt{Set} repräsentieren.
    Das gleiche gilt \emph{nicht} für die zugeordneten Werte: da diese mehrmals auftreten können, werden sie von der \texttt{values()}-Methode als Collection zurückgegeben.
    \begin{lstlisting}[gobble=8,basicstyle=\ttfamily\scriptsize]
        public static void main (String[] args) {
            // [...] map like previous slide

            Set<Integer> keys = map.keySet();
            Collection<String> values = map.values();

            System.out.println(keys);
            // prints: [23, 28, 31]

            System.out.println(values);
            // prints: [bar, foo, bar]
        }
    \end{lstlisting}
\end{frame}

\begin{frame}[fragile]{\texttt{Map.keySet().iterator()}}
    Um mit einem \texttt{Iterator} über eine \texttt{Map} zu iterieren, kann man mit dem \texttt{Iterator} des Schlüssel-Sets arbeiten:
    \begin{lstlisting}[gobble=8,basicstyle=\ttfamily\scriptsize]
        public static void main (String[] args) {

            // [...] map, values like previous slide
            Set<Integer> keys = map.keySet();
            Iterator<Integer> iter = keys.iterator();

            while(iter.hasNext()) {
                System.out.print(map.get(iter.next()) + " ");
            } // prints: bar foo bar

            System.out.println(); // print a line break

            for(Integer i: keys) {
                System.out.print(map.get(i) + " ");
            } // prints: bar foo bar
        }
    \end{lstlisting}
\end{frame}

\begin{frame}[fragile]{\texttt{Map} und \texttt{for-each}}
    Man kann auch mit einem \texttt{for-each} direkt über dem Keyset iterieren und dadurch über die \texttt{Map} iterieren:
    \begin{lstlisting}[gobble=8]
          Map<String, String> map = ...
          for (Map.Entry<String, String> entry : map.entrySet()) {
            System.out.println("Key: " + entry.getKey() +
            ", value" + entry.getValue());
          }
    \end{lstlisting}
\end{frame}

\begin{frame}{Zusammenfassung}
    \begin{center}
        \begin{tabular}{ l | l }
            List & Speichert die Reihenfolge der Elemente \\
                 & Einfach zu iterieren \\
                 & ineffektive Suche\\
            \hline
            Set  & Keine Duplikate \\
                 & Keine bestimmte Reihenfolge \\
                 & Effektives Suchen \\
            \hline
            Map  & Key-Value-Paare \\
                 & Hocheffektive Suche \\
                 & Iteration komplizierter
        \end{tabular}
    \end{center}
\end{frame}

\begin{frame}{Übung}
    \center
    \large{Wähle Dir eine der folgenden Übungen aus: \\
    \vfill
    1. (Generics) Baue die \texttt{NumberBox} aus dem Kurs-Repo(unter sourcefiles/07-collections/exercise) so um, dass sie Generics verwendet. \\
    \vfill
    2. (Collections) Schreibe ein Programm, in dem du eine Implementierung von \texttt{java.util.List} mit Elementen befüllst und diese mittels eines \texttt{Iterators} ausgibst. Welche Methoden werden von diesem \texttt{Iterator} verwendet?}
\end{frame}
\end{document}
