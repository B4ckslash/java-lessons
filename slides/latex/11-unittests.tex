\documentclass[10pt]{beamer}

%theme
\usetheme{metropolis}
% packages
\usepackage{color}
\usepackage{listings}
\usepackage[main=ngerman, USenglish]{babel}
\usepackage[utf8]{inputenc}
\usepackage{multicol}
\usepackage{csquotes}
\usepackage{hyperref}

% color definitions
\definecolor{mygreen}{rgb}{0,0.6,0}
\definecolor{mygray}{rgb}{0.5,0.5,0.5}
\definecolor{mymauve}{rgb}{0.58,0,0.82}
\definecolor{paleorange}{HTML}{CC7832}

\lstset{
    backgroundcolor=\color{white},
    % choose the background color;
    % you must add \usepackage{color} or \usepackage{xcolor}
    basicstyle=\footnotesize\ttfamily,
    % the size of the fonts that are used for the code
    breakatwhitespace=false,
    % sets if automatic breaks should only happen at whitespace
    breaklines=true,                 % sets automatic line breaking
    captionpos=b,                    % sets the caption-position to bottom
    commentstyle=\color{mygreen},    % comment style
    % deletekeywords={...},
    % if you want to delete keywords from the given language
    extendedchars=true,
    % lets you use non-ASCII characters;
    % for 8-bits encodings only, does not work with UTF-8
    frame=single,                    % adds a frame around the code
    keepspaces=true,
    % keeps spaces in text,
    % useful for keeping indentation of code
    % (possibly needs columns=flexible)
    keywordstyle=\color{blue},       % keyword style
    % morekeywords={*,...},
    % if you want to add more keywords to the set
    numbers=left,
    % where to put the line-numbers; possible values are (none, left, right)
    numbersep=5pt,
    % how far the line-numbers are from the code
    numberstyle=\tiny\color{mygray},
    % the style that is used for the line-numbers
    rulecolor=\color{black},
    % if not set, the frame-color may be changed on line-breaks
    % within not-black text (e.g. comments (green here))
    stepnumber=1,
    % the step between two line-numbers.
    % If it's 1, each line will be numbered
    stringstyle=\color{mymauve},     % string literal style
    tabsize=4,                       % sets default tabsize to 4 spaces
    % show the filename of files included with \lstinputlisting;
    % also try caption instead of title
    language = Java,
	showspaces = false,
	showtabs = false,
	showstringspaces = false,
	escapechar = ,
        morecomment=[s][\textcolor{paleorange}]{@}{\ },
}

\def\ContinueLineNumber{\lstset{firstnumber=last}}
\def\StartLineAt#1{\lstset{firstnumber=#1}}
\let\numberLineAt\StartLineAt

\newcommand{\codeline}[1]{
        \alert{\texttt{#1}}
}

% Authors of the slides
\author{Florian Pix}

% Name of the Course
\institute{Java-Kurs}

% Presentation title
\title{Unittests mit Java}
\date{\today}

% Fancy Logo
\titlegraphic{\hfill\includegraphics[height=1.25cm]{../templates/fsr_logo_cropped}}

\begin{document}

\begin{frame}{Unittests mit Java}
    \titlepage
\end{frame}

\begin{frame}{Gliederung}
	\setbeamertemplate{section in toc}[sections numbered]
	\tableofcontents
\end{frame}

\begin{frame}[fragile]{Unittests}
    \section{Unittests}
\end{frame}

\begin{frame}[fragile]{Unittests}
    \textit{Unittests} sind Tests, die Einzelteile eures Programms überprüfen sollen. Sie können Teil ganzer \textit{Integrationstests} sein die das Zusammenspiel der einzelnen Einheiten (Units) 		              untersuchen. \\
\end{frame}

\begin{frame}[fragile]{Unittests}
    \textit{JUnit} ist ein Framework für Java, dass wir zum Testen nutzen können. 
\end{frame}

\begin{frame}[fragile]{Unittests}
    \textit{JUnit} ist ein Framework für Java, dass wir zum Testen nutzen können. 

    Stand 20.Januar 2019, ist JUnit5 die aktuellste Version, \\
    die auch im SWT Praktikum (3.Semester) genutzt wurde.

    \url{https://junit.org/junit5/}
\end{frame}

\begin{frame}[fragile]{Unittests}
    \textit{JUnit} ist ein Framework für Java, dass wir zum Testen nutzen können. 

    Stand 20.Januar 2019, ist JUnit5 die aktuellste Version, \\
    die auch im SWT Praktikum (3.Semester) genutzt wurde.

    \url{https://junit.org/junit5/}
    
    Um es in euer Projekt einzubinden könnt ihr nach dem Download:

    Eclipse: Project Properties\rightarrow Java Build Path \newline Libraries\rightarrow Add External JARs... 

    IntelliJ: File\rightarrow Project Structure\rightarrow Modules \newline Dependencies\rightarrow + \rightarrow JARs or directories
\end{frame}

\begin{frame}[fragile]{Oft genutzte Funktionen}
    \section{Oft genutzte Funktionen}
\end{frame}

\begin{frame}[fragile]{Oft genutzte Funktionen}
    
     \textcolor{mygreen}{\textbf{assertEquals()}} \\
     \textcolor{mygreen}{\textbf{assertTrue()}} \\
     \textcolor{mygreen}{\textbf{assertThrows()}} \\
     \textcolor{mygreen}{\textbf{fail()}}

    \href{{https://junit.org/junit5/docs/current/api/org/junit/jupiter/api/Assertions.html}}{Javadoc Assertions}
\end{frame}

\begin{frame}[fragile]{Oft genutzte Funktionen}
     \textcolor{mygreen}{\textbf{assertEquals()}}

    \begin{lstlisting}[basicstyle=\ttfamily\scriptsize,gobble=8]
        public class AClass{
			AClass(){}
		}	 
		void testCalculate(){
		int expected = 5;
		int actual = AClass.add(2, 3);
        assertEquals(expected, actual); 
        }
    \end{lstlisting}
    Unser erwarteter Wert entspricht dem tatsächlichen, \newline 
    also hat diese Einheit diesen Test bestanden.
\end{frame}

\begin{frame}[fragile]{Oft genutzte Funktionen}
     \textcolor{mygreen}{\textbf{assertTrue()}}

    \begin{lstlisting}[basicstyle=\ttfamily\scriptsize,gobble=8]
        public class AClass{
			boolean noProblems = True;
			AClass(){}
			...
			public importantFunction(...);
		}	
		...
		AClass aClass = new AClass();
		...
		void testCalculate(){	
			AClass.importantFunction(...);
        	assertTrue(AClass.getNoProblems()); 
        }
    \end{lstlisting}
    Wenn importantFunction() unsere Flag nicht ändert,  besteht die Unit den Test.
\end{frame}

\begin{frame}[fragile]{Oft genutzte Funktionen}
     \textcolor{mygreen}{\textbf{assertThrows()}}
    
    Wir wollen testen ob eine Funktion unter bestimmten Bedingungen eine Exception wirft.

    \begin{lstlisting}[basicstyle=\ttfamily\scriptsize,gobble=8]
        assertThrows(AClass.throw());     
    \end{lstlisting}
    Wenn throw() eine Exception wirft, besteht die Einheit den Test.
\end{frame}

\begin{frame}[fragile]{Oft genutzte Funktionen}
     \textcolor{mygreen}{\textbf{fail()}}
    \begin{lstlisting}[basicstyle=\ttfamily\scriptsize,gobble=8]
        void testSomething(){
			int result = AClass.aFunction();
			if(result < 4.0){
				fail("Result should be above 4.0");
			}
		}     
    \end{lstlisting}
    \begin{lstlisting}[basicstyle=\ttfamily\scriptsize,gobble=8]
        void testSomething(){
			try{
				AClass.bFunction();
			} catch (SomethingWentTerriblyWrongException() e) {
				fail("Something went terribly wrong");
			}
		}     
    \end{lstlisting}
    Wenn man beim Durchlaufen des Tests auf ein fail() trifft, schlägt er fehl.
\end{frame}

\begin{frame}[fragile]{Annotations}
    \section{Annotations}
\end{frame}

\begin{frame}[fragile]{Annotations}
    \textcolor{mygreen}{\textbf{@Test}}\\
    \textcolor{mygreen}{\textbf{@BeforeAll, @AfterAll}}\\
    \textcolor{mygreen}{\textbf{@BeforeEach, @AfterEach}}\\
\end{frame}

\begin{frame}[fragile]{Annotations}
     \textcolor{mygreen}{\textbf{@Test}}
    \begin{lstlisting}[basicstyle=\ttfamily\scriptsize,gobble=8]
    	@Test		
    	public void test1() {					
            list.add("test");					
            assertFalse(list.isEmpty());			
            assertEquals(1, list.size());			
    	}
    \end{lstlisting}
\end{frame}

\begin{frame}[fragile]{Annotations}
     \textcolor{mygreen}{\textbf{@BeforeAll, @AfterAll}}
    \begin{lstlisting}[basicstyle=\ttfamily\scriptsize,gobble=8]
    	@BeforeAll		
    	public static void setUpAll() {							
            System.out.println
            ("@BeforeAll , wird vor allen TestCases einmal ausgefuehrt.");					
    	}

    	@AfterAll
    	public static void cleanUpAll() {							
            System.out.println
            ("@AfterAll ,wird nach allen TestCases einmal ausgefuehrt.");					
    	}		
    \end{lstlisting}
\end{frame}

\begin{frame}[fragile]{Annotations}
     \textcolor{mygreen}{\textbf{@BeforeEach, @AfterEach}}
    \begin{lstlisting}[basicstyle=\ttfamily\scriptsize,gobble=8]

    	@BeforeEach		
    	public void setUpEach() {					
        	list = new ArrayList<String>();					
        	System.out.println
                ("@BeforeEach ,wird vor jedem einzelnen TestCase ausgefuehrt.");					
    	}

    	@AfterEach
    	public void cleanUpAfterEach() {					
        	list.clear();			
        	System.out.println
                ("@AfterEach, wird nach jedem einzelnen TestCase ausgefuehrt.");					
    	}				
    \end{lstlisting}
\end{frame}

\begin{frame}[fragile]{Annotations}
    \textcolor{mygreen}{\textbf{@ParameterizedTest,@RepeatedTest}}
    
    \begin{quote}
    Tests können wiederholt ausgeführt werden, sogar mit unterschiedlichen Parametern.
    \end{quote}
    \textcolor{mygreen}{\textbf{@DisplayName}}

    \begin{quote}
    Testklassen und -methoden können eigene Namen mit Leerzeichen, Sonderzeichen und sogar Emojis haben.
    \end{quote}

    \href{https://dzone.com/articles/junit-5-annotations-with-examples-1}{Weitere tolle Annotations}\\
\end{frame}

\begin{frame}[fragile]{Testrunner}
    \section{Testrunner}
\end{frame}

\begin{frame}[fragile]{Testrunner}
    Entweder nutzt ihr den in eurer IDE eingebauten Testrunner, \newline
    oder schreibt euch einen eigenen.

    \begin{lstlisting}[basicstyle=\ttfamily\scriptsize,gobble=8]
        import org.junit.runner.JUnitCore;
        import org.junit.runner.Result;	
        import org.junit.runner.notification.Failure;
        public class TestRunner {
            public static void main(String[] args) {
                Result result = JUnitCore.runClasses
                                     (JunitAnnotationsExample.class);
                for (Failure failure : result.getFailures()) {
                     System.out.println(failure.toString());
                }
                System.out.println("Result = " + result.wasSuccessful());
            }		
        }      	
    \end{lstlisting}
\end{frame}

\begin{frame}[fragile]{Demo und Aufgabe}
    \section{Demo und Aufgabe}
\end{frame}

\end{document}